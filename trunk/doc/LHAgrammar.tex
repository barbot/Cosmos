\documentclass{article}
\usepackage{syntax}
\usepackage{amssymb}
\usepackage{amsmath}

\begin{document}


\section{General Expression}

\begin{grammar}
  <MarkingDepExpr> ::= <ConstIdentifier> | <DiscreteVarIdentifier> | <PlaceIdentifier>
  \alt  <MarkingDepExpr> `+'  <MarkingDepExpr> | <MarkingDepExpr> `-'  <MarkingDepExpr>
  \alt  <MarkingDepExpr> `*'  <MarkingDepExpr>
  \alt `Card' <MarkingDepExpr>
  \alt <MarkingDepExpr> `[' <ColorList> `]'
  \alt <DomainMarking>

 <Expr> ::= <ConstIdentifier> | <DiscreteVarIdentifier>  | <VarIdentifier> | <PlaceIdentifier>
  \alt  <Expr> `+'  <Expr> | <Expr> `-'  <Expr>
  \alt  <Expr> `*'  <Expr>
  \alt `Card' <Expr>
  \alt <Expr> `[' <ColorList> `]'
  \alt <DomainMarking>

  <ColorList> ::= <ColorExpr> `,' <ColorList>
  \alt <ColorExpr>

  <ColorExpr> ::= <ColorIdentifier> | <ColorVarIdentifier>
  \alt <ColorExpr> `++' | <ColorExpr> `--'

  <LinearExpr> ::= <LinearExpr> `+'  <LinearExpr> | <LinearExpr> `-'  <LinearExpr>
  \alt <MarkingDepExpr> `*' <VarExpr>

  <VarExpr> ::= <VarIdentifier> | <VarIdentifier> `[' <ColorList> `]'

  <DiscreteVarExpr> ::= <DiscreteVarExpr> \alt <DiscreteVarExpr> `[' <ColorList> `]'

<MarkingDepProp> ::= `true' | `false' | `not' <MarkingDepProp>
  \alt <MarkingDepProp> `\&\&' <MarkingDepProp> |  <MarkingDepProp> `||' <MarkingDepProp>
  \alt <MarkingDepExpr> `|\!\!>\!<\!\!|' <MarkingDepExpr>  
  \alt <ColorExpr> `=' <ColorExpr> | <ColorExpr> `!=' <ColorExpr>

\end{grammar}

All the expression are typed with a domain. An expression of a domain
$D$ is a function that map to each color of the domain $D$ an integer
or real expression. 

Value of the domain ``Uncolored'' act as scalar.  The construction
``Card'' take an expression over a domain and return an expression of
the uncolored domain by taking the sum of the expression over all
color. The construction ``Expr[c1,c2]'' return a value in the
uncolored domain equal to the valuation of ``Expr'' for the color
``c1,c2''.


There is two type of expression, The one describe
by ``Expr'' can depend of any quantities define in the Net or in the
automaton whereas ``MarkingDepExpr'' can not depend of quantities
which are continuous variable. The difference is that
``MarkingDepExpr'' are constant between two transition of the
automaton whereas ``Expr'' evolve linearly between two
transition. Both of them can be discontinuous when a transition of the
automaton occurs.


\section{Declarative Part and General Expression}

\begin{grammar}
  <Declaration> ::= <VarDeclaration> `;' | <ConstDeclaration> `;'

  <ConstDeclaration> ::= 
  `const' `int' <IDENTIFIER> `=' <INTEGER>
  \alt `const' `double' <IDENTIFIER> `=' <FLOAT>

  <VarDeclaration> ::= 
   `var' <IDENTIFIER> (`in' <DomainIdentifier>) ?
   \alt `discvar' <IDENTIFIER> (`in' <DomainIdentifier>) ?
   \alt `colorvar' <IDENTIFIER> `in' <DomainIdentifier>
\end{grammar}

This section define the identifier for the different type of variable
as well as their domain.

\section{Location Invariant Proposition}

\begin{grammar}
  <LocInvProp> ::= <MarkingDepProp>
\end{grammar}

Location invariant do not depend of continuous variable value.
It ensure that location invariant are constant between two transitions
of the automaton.

\section{Flow Expression}

\begin{grammar}
  <FlowDef> ::= <VarExpr> `\'' `=' <MarkingDepExpr> 
\end{grammar}
As location invariant, flow rate are constant between two transitions
of the automaton to ensure that continuous variables are piecewise
linear.


\section{Guard Proposition}
\begin{grammar}
  <GuardProp> ::= `true' | <GuardProp> `\&\&' <GuardProp>
  \alt <LinearExp> `=' <MarkingDepExpr> \alt <LinExp> `<=' <MarkingDepExpr>
  \alt <LinearExp> `>=' <MarkingDepExpr>
  \alt <MarkingDepProp>

  <ColorExpr> ::= \dots\! | <BindingVarIdentifier>
\end{grammar}


\section{Update Expression}

\begin{grammar}
  <UpdateExpr> ::= <VarExpr> `:=' <Expr> | <DiscreteVarExpr> `:=' <Expr>
  \alt <ColorVarIdentifier> `:=' <ColorExpr>

   <ColorExpr> ::= \dots\! | <BindingVarIdentifier>
\end{grammar}



\end{document}
